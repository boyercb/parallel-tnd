\documentclass{article}

% Language setting
% Replace `english' with e.g. `spanish' to change the document language
\usepackage[english]{babel}

% Set page size and margins
% Replace `letterpaper' with `a4paper' for UK/EU standard size
\usepackage[letterpaper,top=2cm,bottom=2cm,left=1cm,right=1cm,marginparwidth=1.75cm]{geometry}

% Useful packages
\usepackage{natbib}
\usepackage{amsmath,amsthm,amsfonts}
\usepackage{graphicx}
\usepackage[colorlinks=true, allcolors=blue]{hyperref}
\usepackage{bbm}


\usepackage{pgf,tikz} 
\usetikzlibrary{arrows,shapes.arrows,shapes.geometric,shapes.multipart,decorations.pathmorphing,positioning,shapes.swigs}

\def\ll{\lambda}
\def\LL{\Lambda}
\def\arctanh{\mathrm{arctanh}}
\def\tanh{\mathrm{tanh}}
\def\expit{\mathrm{expit}}
\def\indep{\!\perp\!\!\!\perp}
\DeclareMathOperator{\E}{E}

\newtheorem{example}{Example}
\newtheorem{theorem}{Theorem}
\title{New simulation}


\begin{document}
\maketitle

Assume the distribution of the variables follow the SWIG as in Figure 1


\begin{figure}[!htbp]
		\centering
		\begin{tikzpicture}
			
			\tikzset{line width=1pt,inner sep=5pt,
				%	%swig vsplit={gap=3pt, inner line width right=0.4pt},
				ell/.style={draw, inner sep=1.5pt,line width=1pt}}
			
			\node[name=V, shape=swig vsplit]{ \nodepart{left}{$V$} \nodepart{right}{$v$} };

			\node (Vname) at (0, 0.5-1.5) {Vaccination};
			
			
			\node[shape=ellipse,ell] (I) at (2.5,0) {$I^v$};
			\node (Iname) at (2.5,0.5-1.5) {Illness};
			
			
			\node[shape=ellipse,ell] (T) at (5,0) {$T^v$};
			\node (Tname) at (5,0.5-1.5) {Testing};
			
			
			
			\node[shape=ellipse,ell] (U) at (2.5,1.5+1) {$U,X$};
			%			\node (HSname) at (0,1.5+0.5+0.5) {Unmeasured};
			\node (Uname) at (2.5,1.5+0.5+0.5+1) {Measured \& unmeasured};
			\node (Uname2) at (2.5,1.5+0.5+1) {confounding};
			\draw[-stealth, line width = 0.5pt] (U) to (-0.2, 0.31);
                \draw[-stealth, line width=0.5pt, bend right](V) to (T);

   
			\foreach \from/\to in {V/I, I/T, U/I, U/T}
			\draw[-stealth, line width = 0.5pt] (\from) -- (\to);
			%%%NCs
		
					
		\end{tikzpicture}
		\caption{A Single-World Intervention Graph (SWIG) for causal relationship between variables of a test-negative design in the simulation.}
	\end{figure} 

The SWIG implies $I^v, T^v\indep V\mid U, X$ for $v=0,1$. Consider the following setting:


\begin{align*}
    X, U &\sim Unif(0,1)\\
    V\mid U, X & \sim Bernoulli(\expit(\alpha_0 + \alpha_U U + \alpha_X X))\\
    I^v \mid  U, X &\sim Multinomial(1-p_1(X, U, v), p_1(X, U, v), p_2(X, U, v))\\
    T^v\mid I^v,  U, X &\sim Bernoulli(\mathbbm 1(I^v>0)\exp\{\tau_{1}\mathbbm 1(I^v=1) + \tau_{2}\mathbbm 1(I^v=2)+\tau_V v  + \tau_U U + \tau_{2U}\mathbbm 1(I^v=2) U +  \tau_X X\})
\end{align*}

where \begin{align*}
    p_1(X, U, V) &= P(I=1\mid X, U, V)=\exp(\beta_{10} + \beta_{1V}V + \beta_{1X}X + \beta_{1VX}VX + \beta_{1U}U)\\
    p_2(X, U, V) &= P(I=2\mid X, U, V)=\exp(\beta_{20} + \beta_{2V}V + \beta_{2X}X + \beta_{2VX}VX + \beta_{2U}U)\\
\end{align*}
Note that $\beta_{1V}=\beta_{1VX}=0$ so the vaccine has no effect on the risk of cause 1. The distribution of $T$ is set such that only ill subjects will be tested. 

We show that the simulation setting satisfies Assumption (A4).

For $i=1,2$ and $v,v'=0,1$, we have
\begin{align*}
    & P(I^v=i,T^v=1\mid V=v', X)\\
    =& E\{P(I^v=i,T^v=1\mid V=v, U, X)\mid V=v', X\}\\
    =& E\{\exp(\beta_{i0}+\beta_{iV}v+ \beta_{iX}X+\beta_{iVX}vX + \beta_{iU}U +\tau_i + \tau_V v + \tau_U U +  \tau_{2U}\mathbbm 1(i=2) U + \tau_X X)\mid V=v', X\}\\
    =& \exp(\beta_{i0} + \beta_{iV}v  +\beta_{iX}X +\beta_{iVX}vX + \tau_i + \tau_V v  + \tau_X X)E\{\exp(\tau_U U+  \tau_{2U}\mathbbm 1(i=2) U +\beta_{iU}U)\mid V=v', X\}.
\end{align*}

Therefore
\begin{align*}
&\dfrac{P(I^0=2, T^0=1\mid V=1,X)}{P(I^0=2, T^0=1 \mid V=0,X)}=\dfrac{E\{\exp(\tau_U U+ \tau_{2U} U 
 + \beta_{2U}U)\mid V=1, X\}}{E\{\exp(\tau_U U+\tau_{2U} U + \beta_{2U}U)\mid V=0, X\}},\\ 
 &\dfrac{P(I^0=1, T^0=1\mid V=1,X)}{P(I^0=1, T^0=1\mid V=0,X)}=\dfrac{E\{\exp(\tau_U U+\beta_{1U}U)\mid V=1, X\}}{E\{\exp(\tau_U U+\beta_{1U}U)\mid V=0, X\}}
\end{align*}
and Assumption (A4) holds if $\beta_{2U}=\beta_{1U}$ and $\tau_{2U}=0$.

We further verify Assumption A5. For $i=1, 2$,
\begin{align*}
    & Pr(T^v=1\mid I^v=i, V=1, X) \\&= E\{Pr(T^v=1\mid I^v=i, V=1, U, X)\mid I^v=i, V=1, X\}\\
    &= E\{Pr(T^v=1\mid I^v=i,  U, X)\mid I^v=i, V=1, X\}\\
    &= E[\exp\{\tau_{i}+\tau_V v  + \tau_U U + \tau_{2U}\mathbbm 1(i=2) U +  \tau_X X\}\mid I^v=i, V=1, X]\\
    &= \exp\{\tau_{i}+\tau_V v +  \tau_X X\}E[\exp\{\tau_U U + \tau_{2U}\mathbbm 1(i=2) U\}\mid I^v=i, V=1, X]\\
    &= \exp\{\tau_{i}+\tau_V v + \tau_X X\}\int \exp\{\tau_U u + \tau_{2U}\mathbbm 1(i=2) u \}f(u\mid I^v=i, V=1, X)du\\
    &= \exp\{\tau_{i}+\tau_V v +  \tau_X X\}\int \exp\{\tau_U u + \tau_{2U}\mathbbm 1(i=2) u \}\times\\
    &\qquad \dfrac{Pr(I^v=i\mid V=1, U=u, X)Pr(V=1\mid U=u, X)f(u\mid X)}{\int Pr(I^v=i\mid V=1, U=s, X)Pr(V=1\mid U=s, X)f(s\mid X) ds}du\\
\end{align*}

Hence
\begin{align*}
    Pr(T^1=1\mid I^1=i, V=1, X) &= \exp\{\tau_{i}+\tau_V +  \tau_X X\}\int \exp\{\tau_U u + \tau_{2U}\mathbbm 1(i=2) u \}\times\\
    &\qquad \dfrac{Pr(I^1=i\mid V=1, U=u, X)Pr(V=1\mid U=u, X)f(u\mid X)}{\int Pr(I^1=i\mid V=1, U=s, X)Pr(V=1\mid U=s, X)f(s\mid X) ds}du\\
    &= \exp\{\tau_{i}+\tau_V  +  \tau_X X\}\int \exp\{\tau_U u + \tau_{2U}\mathbbm 1(i=2) u \}\times\\
    &\qquad \dfrac{\exp(\beta_{iV} + \beta_{iVX}X)Pr(I^0=i\mid V=1, U=u, X)Pr(V=1\mid U=u, X)f(u\mid X)}{\exp(\beta_{iV} + \beta_{iVX}X)\int Pr(I^0=i\mid V=1, U=s, X)Pr(V=1\mid U=s, X)f(s\mid X) ds}du\\
    &= \exp\{\tau_{i}+\tau_V + \tau_X X\}\int \exp\{\tau_U u + \tau_{2U}\mathbbm 1(i=2) u \}\times\\
    &\qquad \dfrac{Pr(I^0=i\mid V=1, U=u, X)Pr(V=1\mid U=u, X)f(u\mid X)}{\int Pr(I^0=i\mid V=1, U=s, X)Pr(V=1\mid U=s, X)f(s\mid X) ds}du
\end{align*}

and 
\begin{align*}
    Pr(T^0=1\mid I^0=i, V=1, X) &= \exp\{\tau_{i} +  \tau_X X\}\int \exp\{\tau_U u + \tau_{2U}\mathbbm 1(i=2) u \}\times\\
    &\qquad \dfrac{Pr(I^0=i\mid V=1, U=u, X)Pr(V=1\mid U=u, X)f(u\mid X)}{\int Pr(I^0=i\mid V=1, U=s, X)Pr(V=1\mid U=s, X)f(s\mid X) ds}du
\end{align*}
So Assumption A5 holds if $\tau_V = 0$.


Finally, the probability of having the illness of interest in the study sample is
\begin{align*}
    &P(I=2\mid V, U, X, T=1)\\
    = &\dfrac{P(I=2\mid V, U, X)P(T=1\mid I=2, V, U, X)}{P(I=2\mid V, U, X)P(T=1\mid I=2, V, U, X) + P(I=1\mid V, U, X)P(T=1\mid I=1, V, U, X)}\\
    =& \exp(\beta_{20}+\beta_{2V}V+\beta_{2U}U + \beta_{2X}X + \beta_{2VX}VX)\exp(\tau_2 + \tau_V V + \tau_U U + \tau_X X)/\\
    &\quad \{ \exp(\beta_{20}+\beta_{2V}V+\beta_{2U}U + \beta_{2X}X + \beta_{2VX}VX)\exp(\tau_2 + \tau_V V + \tau_U U + \tau_X X) +\\
    &\quad \exp(\beta_{10}+\beta_{1U}U + \beta_{1X}X)\exp(\tau_1 + \tau_V V + \tau_U U + \tau_X X)\}\\
    =& \expit\{(\beta_{20}-\beta_{10}+\tau_2 - \tau_1) + \beta_{2V}V+(\beta_{2X}-\beta_{1X})X+\beta_{2VX}VX\},
\end{align*}
which follows a logistic regression model that does not depend on $U$ ($\beta_{1U}=\beta_{2U}$).

On the other hand, we can show that, for $i=1,2$,
\begin{align*}
    & P(V=1\mid I=i, U, X, T=1)= \expit\{(\alpha_0 + \tau_V) + \alpha_U U + (\alpha_X + \beta_{iVX})X\}
\end{align*}
and therefore $P(V=1\mid I, X, T=1)$ does not correspond to any familiar regression models.

In the above data generating mechanism, the risk ratio among the tested is

\begin{align*}
    \Psi_{RRV} &= \dfrac{Pr[I^1=2\mid V=1]}{Pr[I^0=2\mid V=1]}\\
    &= \dfrac{E[Pr\{I^1=2\mid V=1, U, X\}\mid V=1]}{E[Pr\{I^0=2\mid V=1, U, X\}\mid V=1]}\\
    &= \dfrac{E[Pr\{I=2\mid V=1, U, X\}\mid V=1]}{E[Pr\{I=2\mid V=0, U, X\}\mid V=1]}\\
    &= \dfrac{E\{\exp(\beta_{20} + \beta_{2V} + \beta_{2X}X + \beta_{2VX}X + \beta_{2U}U)\mid V=1\}}{E\{\exp(\beta_{20} +  \beta_{2X}X  + \beta_{2U}U)\mid V=1\}}\\
\end{align*}



\end{document}