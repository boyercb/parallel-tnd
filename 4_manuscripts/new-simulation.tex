\documentclass{article}

% Language setting
% Replace `english' with e.g. `spanish' to change the document language
\usepackage[english]{babel}

% Set page size and margins
% Replace `letterpaper' with `a4paper' for UK/EU standard size
\usepackage[letterpaper,top=2cm,bottom=2cm,left=1cm,right=1cm,marginparwidth=1.75cm]{geometry}

% Useful packages
\usepackage{natbib}
\usepackage{amsmath,amsthm,amsfonts}
\usepackage{graphicx}
\usepackage[colorlinks=true, allcolors=blue]{hyperref}
\usepackage{bbm}


\usepackage{pgf,tikz} \usetikzlibrary{arrows,shapes.arrows,shapes.geometric,shapes.multipart, decorations.pathmorphing,positioning,shapes.swigs,}

\def\ll{\lambda}
\def\LL{\Lambda}
\def\arctanh{\mathrm{arctanh}}
\def\tanh{\mathrm{tanh}}
\def\expit{\mathrm{expit}}
\def\indep{\!\perp\!\!\!\perp}
\DeclareMathOperator{\E}{E}

\newtheorem{example}{Example}
\newtheorem{theorem}{Theorem}
\title{New simulation}


\begin{document}
\maketitle

Assume the distribution of the variables follow the SWIG as in Figure 1


\begin{figure}[!htbp]
		\centering
		\begin{tikzpicture}
			
			\tikzset{line width=1pt,inner sep=5pt,
				%	%swig vsplit={gap=3pt, inner line width right=0.4pt},
				ell/.style={draw, inner sep=1.5pt,line width=1pt}}
			
			\node[name=V, shape=swig vsplit]{ \nodepart{left}{$V$} \nodepart{right}{$v$} };

			\node (Vname) at (0, 0.5-1.5) {Vaccination};
			
			
			\node[shape=ellipse,ell] (I) at (2.5,0) {$I^v$};
			\node (Iname) at (2.5,0.5-1.5) {Illness};
			
			
			\node[shape=ellipse,ell] (T) at (5,0) {$T^v$};
			\node (Tname) at (5,0.5-1.5) {Testing};
			
			
			
			\node[shape=ellipse,ell] (U) at (2.5,1.5+1) {$U,X$};
			%			\node (HSname) at (0,1.5+0.5+0.5) {Unmeasured};
			\node (Uname) at (2.5,1.5+0.5+0.5+1) {Measured \& unmeasured};
			\node (Uname2) at (2.5,1.5+0.5+1) {confounding};
			\draw[-stealth, line width = 0.5pt] (U) to (-0.2, 0.31);
                \draw[-stealth, line width=0.5pt, bend right](V) to (T);

   
			\foreach \from/\to in {V/I, I/T, U/I, U/T}
			\draw[-stealth, line width = 0.5pt] (\from) -- (\to);
			%%%NCs
		
					
		\end{tikzpicture}
		\caption{A Single-World Intervention Graph (SWIG) for causal relationship between variables of a test-negative design in the simulation.}
	\end{figure} 

The SWIG implies $I^v, T^v\indep V\mid U, X$ for $v=0,1$. Consider the following setting:


\begin{align*}
    X, U &\sim Unif(0,1)\\
    V\mid U, X & \sim Bernoulli(\expit(\alpha_0 + \alpha_U U + \alpha_X X))\\
    I \mid V, U, X &\sim Multinomial(1-p_1(X, U, V), p_1(X, U, V), p_2(X, U, V))\\
    T\mid I, V,  U, X &\sim Bernoulli(\mathbbm 1(I>0)\exp\{\tau_{1}\mathbbm 1(I=1) + \tau_{2}\mathbbm 1(I=2)+\tau_V V + \tau_U U + \tau_X X\})
\end{align*}

where \begin{align*}
    p_1(X, U, V) &= P(I=1\mid X, U, V)=\exp(\beta_{10} + \beta_{1V}V + \beta_{1X}X + \beta_{1U}U)\\
    p_2(X, U, V) &= P(I=2\mid X, U, V)=\exp(\beta_{20} + \beta_{2V}V + \beta_{2X}X + \beta_{2U}U)\\
\end{align*}
Note that $\beta_{1V}=0$ so the vaccine has no effect on the risk of cause 1. The distribution of $T$ is set such that only ill subjects will be tested. 

We show that the simulation setting satisfies Assumption (A4).

For $i=1,2$ and $v,v'=0,1$, we have
\begin{align*}
    & P(I^v=i,T^v=1\mid V=v', X)\\
    =& E\{P(I^v=i,T^v=1\mid V=v, U, X)\mid V=v', X\}\\
    =& E\{\exp(\beta_{i0}+\beta_{iV}v+\beta_{iU}U + \beta_{iX}X+\tau_i + \tau_V v + \tau_U U + \tau_X X)\mid V=v', X\}\\
    =& \exp(\beta_{i0} + \beta_{iV}v + \beta_{iX}X + \tau_i + \tau_V v  + \tau_X X)E\{\exp(\tau_U U+\beta_{iU}U)\mid V=v', X\}.
\end{align*}

Therefore
\begin{align*}
\dfrac{P(I^0=2\mid V=1,X)}{P(I^0=2\mid V=0,X)}=\dfrac{E\{\exp(\tau_U U+\beta_{2U}U)\mid V=1, X\}}{E\{\exp(\tau_U U+\beta_{2U}U)\mid V=0, X\}}, \quad \dfrac{P(I^0=1\mid V=1,X)}{P(I^0=1\mid V=0,X)}=\dfrac{E\{\exp(\tau_U U+\beta_{1U}U)\mid V=1, X\}}{E\{\exp(\tau_U U+\beta_{1U}U)\mid V=0, X\}}
\end{align*}
and Assumption (A4a) holds if $\beta_{2U}=\beta_{1U}$.


Finally, the probability of having the illness of interest in the study sample is
\begin{align*}
    &P(I=2\mid V, U, X, T=1)\\
    = &\dfrac{P(I=2\mid V, U, X)P(T=1\mid I=2, V, U, X)}{P(I=2\mid V, U, X)P(T=1\mid I=2, V, U, X) + P(I=1\mid V, U, X)P(T=1\mid I=1, V, U, X)}\\
    =& \exp(\beta_{20}+\beta_{2V}V+\beta_{2U}U + \beta_{2X}X)\exp(\tau_2 + \tau_V V + \tau_U U + \tau_X X)/\\
    &\quad \{ \exp(\beta_{20}+\beta_{2V}V+\beta_{2U}U + \beta_{2X}X)\exp(\tau_2 + \tau_V V + \tau_U U + \tau_X X) +\\
    &\quad \exp(\beta_{10}+\beta_{1U}U + \beta_{1X}X)\exp(\tau_1 + \tau_V V + \tau_U U + \tau_X X)\}\\
    =& \expit\{(\beta_{20}-\beta_{10}+\tau_2 - \tau_1) + \beta_{2V}V+(\beta_{2X}-\beta_{1X})X\},
\end{align*}
which follows a logistic regression model that does not depend on $U$ ($\beta_{1U}=\beta_{2U}$).

\bibliographystyle{plainnat}
\bibliography{sample}

\end{document}