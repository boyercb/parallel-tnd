
\begin{appendix}

    \renewcommand{\thefigure}{A\arabic{figure}}
    \setcounter{figure}{0}
    
    \renewcommand{\thetable}{A\arabic{table}}
    \setcounter{table}{0}
    
    \renewcommand{\theequation}{A\arabic{equation}}
    \setcounter{equation}{0}

%    \appendixwithtoc
    \newpage

    \section{Appendix}
    
    \subsection{Proof}
    Consider the conditional risk ratio for the effect of vaccination among the vaccinated, i.e.
    \begin{equation*}
        \psi(X) \equiv \dfrac{\Pr[I^1 = 2, T^1 = 1 | V = 1, X]}{\Pr[I^0 = 2, T^0 = 1 | V = 1, X]}
    \end{equation*}
    Under the consistency assumption (A1) the numerator is equal to $\Pr[I = 2, T = 1 | V = 1, X]$. Focusing on the denominator, under equi-confounding (A3)
    \begin{equation*}
    \Pr[I^0 = 2, T^0 = 1  | V = 1, X] = \frac{\Pr[I^0 = 1, T^0 = 1  | V = 1, X]}{\Pr[I^0 = 1, T^0 = 1  | V = 0, X]}\Pr[I^0 = 2, T^0 = 1 | V = 0, X]
    \end{equation*}
and then by (A1) and (A2) with (A4) ensuring overlap
    \begin{equation*}
     \Pr[I^0 = 2, T^0 = 1  | V = 1, X] = \frac{\Pr[I = 1, T = 1  | V = 1, X]}{\Pr[I = 1, T = 1  | V = 0, X]}\Pr[I = 2, T = 1 | V = 0, X]
    \end{equation*}
Plugging back into the expression for $\psi(X)$, we find the following identifying expression 
    \begin{equation*}
         \phi(X) \equiv \dfrac{\dfrac{\Pr[I = 2, T = 1 | V = 1, X]}{\Pr[I = 1, T = 1 | V = 1, X]}}{\dfrac{\Pr[I = 2, T = 1 | V = 0, X]}{\Pr[I = 1, T = 1 | V = 0, X]}}
    \end{equation*}
which is now strictly among the observables. A key insight is that $\frac{\Pr[I = 1, T =1  | V = 0, X]}{\Pr[I = 1, T = 1 | V = 1, X]}$ acts as a proxy for $\frac{\Pr[I^0 = 2, T =1  | V = 0, X]}{\Pr[I^0 = 2, T = 1 | V = 1, X]}$ essentially a ``parallel trend'' for $I=2$ in absence of vaccination.

Recall that we only observe biased samples from $(X_i, V_i, S_i = 1, I^*_i)$ where $S = I(I \neq 0, T = 1)$, i.e. we only observe test results among the symptomatic and tested. However, we can show that 
    \begin{equation}
         \phi(X) = \dfrac{\dfrac{\Pr[I^* = 1 | S = 1, V = 1, X]}{\Pr[I^* = 0 | S = 1, V = 1, X]}}{\dfrac{\Pr[I^* = 1 | S = 1, V = 0, X]}{\Pr[I^* = 0 | S = 1, V = 0, X]}}
    \end{equation}    
which is the odds ratio comparing odds of testing positive for vaccinated and unvaccinated among the tested only.

\begin{align*}
    \psi_{rrv}(X) &=\underbrace{\frac{\Pr[I = 2, T = 1 | V = 1, X]}{\Pr[I = 2, T = 1 | V = 0, X]}}_{\text{observed risk ratio}} \times \underbrace{\frac{\Pr[I = 2 | V = 0, X]}{\Pr[I^0 = 2 | V = 1, X]}}_{\text{de-biasing confounding}} \times \underbrace{\frac{\Pr[T = 1 | I = 2, V = 0, X]}{\Pr[T^0 = 1 | I^0 = 2, V = 1, X]}}_{\text{de-biasing selection}} 
\end{align*}

\subsection{What if mutual exclusivity is violated?}
Define $I_1$ as the event that an individual is infected with an alternative pathogen and $I_2$ as the event that an individual is infected with the test-positive pathogen. 

\subsection{What if symptom screen is imperfect?}

\subsection{What if there is a direct effect of vaccination on testing behavior?}

\subsection{What if test is imperfect?}
Recall that 

$$\Pr[I^* = 1 | S = 1, V = 1, X] = \Pr[I = 2 | S = 1, V = 1, X]$$

$I^* = \lambda_{sens} \cdot \mathbb{I}(I = 2) + (1 - \lambda_{spec}) \cdot \mathbb{I}(I \neq 2)$
\end{appendix}

