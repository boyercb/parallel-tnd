    \begin{figure}[p]
        \centering
        \begin{subfigure}{0.49\linewidth}
            \centering
            \begin{tikzpicture}[> = stealth, shorten > = 1pt, auto, node distance = 1.75cm, inner sep = 0pt,minimum size = 0.5pt, semithick]
                \tikzstyle{every state}=[
                  draw = none,
                  fill = none
                ]
                \node[state] (x) {$X$};
                \node[state] (v) [right of=x] {$V$};
                \node[state] (i) [right of=v] {$I$};
                \node[state] (t) [right of=i] {$T$};
                \node[state] (is) [right of=t] {$I^*$};
                \node[state] (u) [below of=v] {$U$};
        
                \path[->] (x) edge node {} (v);
                \path[->] (x) edge [out=45, in=135] node {} (i);
                \path[->] (x) edge [out=45, in=135] node {} (t);
                
                \path[->] (v) edge node {} (i);
                
                \path[->] (i) edge node {} (t);
                \path[->] (i) edge [out=45, in=135] node {} (is);
        
                \path[->] (t) edge node {} (is);
        
                \path[->] (u) edge node {} (x);
                \path[->] (u) edge node {} (i);
                \path[->] (u) edge node {} (t);
                \end{tikzpicture}
            \caption{Unconfoundedness assumption\label{fig:dag_a}}
        \end{subfigure}
        \begin{subfigure}{0.49\linewidth}
            \centering
            \begin{tikzpicture}[> = stealth, shorten > = 1pt, auto, node distance = 1.75cm, inner sep = 0pt,minimum size = 0.5pt, semithick]
                \tikzstyle{every state}=[
                  draw = none,
                  fill = none
                ]
                \node[state] (x) {$X$};
                \node[state] (v) [right of=x] {$V$};
                \node[state] (i) [right of=v] {$I$};
                \node[state] (t) [right of=i] {$\boxed{T}$};
                \node[state] (is) [right of=t] {$I^*$};
                \node[state] (u) [below of=v] {$\boxed{U}$};
        
                \path[->] (x) edge node {} (v);
                \path[->] (x) edge [out=45, in=135] node {} (i);
                \path[->] (x) edge [out=45, in=135] node {} (t);
                
                \path[->] (v) edge node {} (i);
                
                \path[->] (i) edge node {} (t);
                \path[->] (i) edge [out=45, in=135] node {} (is);
        
                \path[->] (t) edge node {} (is);
        
                \path[->] (u) edge node {} (x);
                \path[->] (u) edge node {} (i);
                \path[->] (u) edge node {} (v);
                \path[-] (u) edge [double, thick, double distance=1pt] node {} (t);
                \end{tikzpicture}
            \caption{Testing synonymous with unmeasured confounding\label{fig:dag_b}}
        \end{subfigure}
        \begin{subfigure}{0.49\linewidth}
        \centering
        \begin{tikzpicture}[> = stealth, shorten > = 1pt, auto, node distance = 1.75cm, inner sep = 0pt,minimum size = 0.5pt, semithick]
            \tikzstyle{every state}=[
              draw = none,
              fill = none
            ]
            \node[state] (x) {$X$};
            \node[state] (v) [right of=x] {$V$};
            \node[state] (i) [right of=v] {$I$};
            \node[state] (t) [right of=i] {$T$};
            \node[state] (is) [right of=t] {$I^*$};
            \node[state] (u) [below of=v] {$U$};
    
            \path[->] (x) edge node {} (v);
            \path[->] (x) edge [out=45, in=135] node {} (i);
            \path[->] (x) edge [out=45, in=135] node {} (t);
            
            \path[->] (v) edge node {} (i);
            
            \path[->] (i) edge node {} (t);
            \path[->] (i) edge [out=45, in=135] node {} (is);
    
            \path[->] (t) edge node {} (is);
    
            \path[->] (u) edge node {} (x);
            \path[->] (u) edge node {} (v);
            \path[->] (u) edge [line width=2pt] node {} (i);
            \path[->] (u) edge [line width=2pt] node {} (t);
    
    
            % \path[->] (x) edge [dashed] node {} (v);
            % \path[->] (x) edge [dashed] node {} (e);
            % \path[->] (x) edge [dashed] node {} (i);
            % \path[->] (x) edge [dashed] node {} (h);
            % \path[->] (x) edge [dashed] node {} (d);
            
            \end{tikzpicture}
        \caption{Equi-confounding and equi-selection assumptions\label{fig:dag_c}}
        \end{subfigure}
        \begin{subfigure}{0.49\linewidth}
            \centering
            \begin{tikzpicture}[> = stealth, shorten > = 1pt, auto, node distance = 1.75cm, inner sep = 0pt,minimum size = 0.5pt, semithick]
                \tikzstyle{every state}=[
                  draw = none,
                  fill = none
                ]
                \node[state] (x) {$X$};
                \node[state] (v) [right of=x] {$V$};
                \node[state] (i) [right of=v] {$I$};
                \node[state] (t) [right of=i] {$T$};
                \node[state] (is) [right of=t] {$I^*$};
                \node[state] (u) [below of=v] {$U$};
        
                \path[->] (x) edge node {} (v);
                \path[->] (x) edge [out=45, in=135] node {} (i);
                \path[->] (x) edge [out=45, in=135] node {} (t);
                
                \path[->] (v) edge node {} (i);
                
                \path[->] (i) edge node {} (t);
                \path[->] (i) edge [out=45, in=135] node {} (is);
        
                \path[->] (t) edge node {} (is);
        
                \path[->] (u) edge node {} (x);
                \path[->] (u) edge node {} (v);
                \path[->] (u) edge [line width=2pt] node {} (i);
                % \path[->] (u) edge [line width=2pt] node {} (t);
        
        
                % \path[->] (x) edge [dashed] node {} (v);
                % \path[->] (x) edge [dashed] node {} (e);
                % \path[->] (x) edge [dashed] node {} (i);
                % \path[->] (x) edge [dashed] node {} (h);
                % \path[->] (x) edge [dashed] node {} (d);
                
                \end{tikzpicture}
            \caption{Equi-confounding only\label{fig:dag_d}}
        \end{subfigure}
        \begin{subfigure}{0.49\linewidth}
        \centering
        \begin{tikzpicture}[> = stealth, shorten > = 1pt, auto, node distance = 1.75cm, inner sep = 0pt,minimum size = 0.5pt, semithick]
            \tikzstyle{every state}=[
              draw = none,
              fill = none
            ]
            \node[state] (x) {$X$};
            \node[state] (v) [right of=x] {$V$};
            \node[state] (i) [right of=v] {$I$};
            \node[state] (t) [right of=i] {$T$};
            \node[state] (is) [right of=t] {$I^*$};
            \node[state] (u) [below of=v] {$U$};
    
            \path[->] (x) edge node {} (v);
            \path[->] (x) edge [out=45, in=135] node {} (i);
            \path[->] (x) edge [out=45, in=135] node {} (t);
            
            \path[->] (v) edge node {} (i);
            
            \path[->] (i) edge node {} (t);
            \path[->] (i) edge [out=45, in=135] node {} (is);
    
            \path[->] (t) edge node {} (is);
    
            \path[->] (u) edge node {} (x);
            \path[->] (u) edge node {} (v);
            % \path[->] (u) edge [line width=2pt] node {} (i);
            \path[->] (u) edge [line width=2pt] node {} (t);
    
    
            % \path[->] (x) edge [dashed] node {} (v);
            % \path[->] (x) edge [dashed] node {} (e);
            % \path[->] (x) edge [dashed] node {} (i);
            % \path[->] (x) edge [dashed] node {} (h);
            % \path[->] (x) edge [dashed] node {} (d);
            
            \end{tikzpicture}
        \caption{Equi-selection only\label{fig:dag_e}}
        \end{subfigure}
        \caption{Causal directed acyclic graphs for the test-negative design under alternative assumption sets. Graph (a) shows a mechanism satisfying the unconfoundedness assumption discussed in \cite{schnitzer_estimands_2022} as, conditional on $X$, $V$ and $I$ and $T$ are d-separated. Graph (b) shows a mechanism discussed in \cite{sullivan_theoretical_2016} where $U$ is binary health seeking behavior that is synonymous with receiving a test and therefore effectively conditioned on by conditioning on testing. Graph (c) shows a possible mechanism satisfying the identifiability conditions in Section \ref{sec:conditions} of this paper. It includes the possibility of unmeasured confounding and selection bias by an arbitrary $U$, which could be nonbinary; however, as shown by the bold arrows $U$ is assumed to act equivalently, on the odds scale, for the test positive and test negative illnesses. Graphs (d) and (e) show alternative mechanisms that satisfy equi-confounding or equi-selection only.}\label{fig:dags}
    \end{figure}